% !TeX encoding = UTF-8
% !TeX program = pdflatex

%----------------------------------------------------------------------------------------
%    PACKAGES AND OTHER DOCUMENT CONFIGURATIONS
%----------------------------------------------------------------------------------------

\documentclass[paper=a4, fontsize=11pt]{scrartcl} % A4 paper and 11pt font size

\usepackage[utf8]{inputenc}
\usepackage{hyperref}
\usepackage{graphicx}
\usepackage{listings}
\usepackage[dvipsnames]{xcolor}   % for \textcolor
\usepackage{relsize}

\usepackage[T1]{fontenc} % Use 8-bit encoding that has 256 glyphs
%\usepackage{fourier} % Use the Adobe Utopia font for the document - comment this line to return to the LaTeX default
%\usepackage[english]{babel} % English language/hyphenation
\usepackage{amsmath,amsfonts,amsthm,amssymb} % Math packages

\usepackage{sectsty} % Allows customizing section commands
\allsectionsfont{\centering \normalfont\scshape} % Make all sections centered, the default font and small caps

\usepackage{fancyhdr} % Custom headers and footers
\pagestyle{fancyplain} % Makes all pages in the document conform to the custom headers and footers
\fancyhead{} % No page header - if you want one, create it in the same way as the footers below
\fancyfoot[L]{} % Empty left footer
\fancyfoot[C]{} % Empty center footer
\fancyfoot[R]{\thepage} % Page numbering for right footer
\renewcommand{\headrulewidth}{0pt} % Remove header underlines
\renewcommand{\footrulewidth}{0pt} % Remove footer underlines
\setlength{\headheight}{13.6pt} % Customize the height of the header

\numberwithin{equation}{section} % Number equations within sections (i.e. 1.1, 1.2, 2.1, 2.2 instead of 1, 2, 3, 4)
\numberwithin{figure}{section} % Number figures within sections (i.e. 1.1, 1.2, 2.1, 2.2 instead of 1, 2, 3, 4)
\numberwithin{table}{section} % Number tables within sections (i.e. 1.1, 1.2, 2.1, 2.2 instead of 1, 2, 3, 4)

\setlength\parindent{0pt} % Removes all indentation from paragraphs - comment this line for an assignment with lots of text

\DeclareMathOperator*{\E}{\mathbb{E}}

\hypersetup{
    colorlinks,
    citecolor=black,
    filecolor=black,
    linkcolor=black,
    urlcolor=black
}

\lstnewenvironment{pycode}[1][]{
    \lstset{
        language=python,
        numbers=left,
        stepnumber=1,
        breaklines=true,
        basicstyle=\linespread{1.0}\ttfamily,
        keywordstyle=\color{blue}\ttfamily,
        stringstyle=\color{red}\ttfamily,
        commentstyle=\color{OliveGreen}\ttfamily,
        morecomment=[l][\color{magenta}]{\#}
        columns=fullflexible,
        otherkeywords={self,as},
        postbreak=\mbox{\textcolor{red}{$\hookrightarrow$}\space},
        escapeinside={(*@}{@*)},
        showstringspaces=false,
    }
    \lstdefinestyle{nonumbers}
    {numbers=none}
}{}

\lstnewenvironment{ccode}[1][]{
    \lstset{
        language=C,
        numbers=left,
        stepnumber=1,
        breaklines=true,
        basicstyle=\linespread{1.0}\ttfamily,
        keywordstyle=\color{blue}\ttfamily,
        stringstyle=\color{red}\ttfamily,
        commentstyle=\color{OliveGreen}\ttfamily,
        morecomment=[l][\color{magenta}]{\#}
        columns=fullflexible,
        otherkeywords={},
        postbreak=\mbox{\textcolor{red}{$\hookrightarrow$}\space},
        escapeinside={(*@}{@*)},
        showstringspaces=false,
    }
    \lstdefinestyle{nonumbers}
    {numbers=none}
}{}

\lstdefinelanguage{pseudocode}{
    alsodigit = {-},
    keywords = {algorithm,new,if,else,while,for,foreach,to,do,in,return,true,false,and,or,not,error,is,skip}
}


\lstnewenvironment{pseudo}[1][]{
    \lstset{
        language=pseudocode,
        mathescape=true,
        numbers=left,
        stepnumber=1,
        breaklines=true,
        basicstyle=\linespread{1.0},
        keywordstyle=\bfseries,
        stringstyle=\color{red}\ttfamily,
        columns=fullflexible,
        otherkeywords={},
        postbreak=\mbox{\textcolor{red}{$\hookrightarrow$}\space},
        showstringspaces=false,
    }
    \lstdefinestyle{nonumbers}
    {numbers=none}
}{}


\makeatletter
\def\BState{\State\hskip-\ALG@thistlm}
\makeatother


\newcommand{\horrule}[1]{\rule{\linewidth}{#1}} % Create horizontal rule command with 1 argument of height

\setcounter{secnumdepth}{0}

\title{    
\normalfont \normalsize 
\textsc{Sapienza University of Rome} \\ [25pt] % Your university, school and/or department name(s)
\horrule{0.5pt} \\[0.4cm] % Thin top horizontal rule
\huge Algorithm Design Second Homework \\ % The assignment title
\horrule{2pt} \\[0.5cm] % Thick bottom horizontal rule
}

\author{Andrea Fioraldi 1692419} % Your name

\date{\normalsize\today} % Today's date or a custom date

\begin{document}

\maketitle % Print the title

\section{Exercise 1}

We redefine our problem as a problem on graphs. Let $G(M, F, E)$ a fully connected graph in which each friend is a node. Each edge $(u,v)$ as a cost $w(u,v) \geq 0$.
We define the following functions:

$deg(v, G) := \sum_{u \in G.M \cup G.F} w(v, u)$;

$density(G) := \frac{1}{|C.M \cup G.F|}\sum_{v, u \in G.M \cup G.F} w(v, u)$;

\begin{pseudo}
algorithm $michele\_party\_approx(G)$
    $n := |G.M \cup G.F|$
    $H_n := G$
    $max\_d := -1$
    $s := -1$
    for $i = \frac{n}{2}$ to $1$
        $d := density(H_i)$
        if $d > max\_d$
            $max\_d = d$
            $s = i$
        $m := v \in H_i.M\text{ }|\text{ }deg(v, H_i) \leq deg(u, H_i)\text{ }\forall u \in H_i.M$
        $f := v \in H_i.F\text{ }|\text{ }deg(v, H_i) \leq deg(u, H_i)\text{ }\forall u \in H_i.F$
        $H_{i-1} = H_i \setminus \{m, f\}$
    return $H_s$
\end{pseudo}

Let $OPT$ the optimal solution and $d_{OPT} = \frac{1}{|OPT|}\sum_{v, u \in OPT} w(v, u) = \frac{W}{N}$ its density. Let $m \in OPT \cap M$ and $f \in OPT \cap F$.

Consider that $\frac{W}{N} \geq \frac{W-deg(m)-deg(f)}{N-2} \Leftrightarrow \frac{deg(m)+deg(f)}{N-2} \geq \frac{W}{N-2} - \frac{W}{N} \Leftrightarrow deg(m)+deg(f) \geq W - \frac{N-2}{N}*W = 2*\frac{W}{N}$.
So in $OPT$ the sum of the degree of a generic $m$ and $f$ is at least twice $d_{OPT}$.

\bigskip
Consider the iteration $i$ of the algorithm in which for the first time two nodes $m \in OPT \cap M$ and $f \in OPT \cap F$ are removed. 
In the $H_i$ graph all the pairs $m \in H_i \cap M$ and $f \in H_i \cap F$ have $deg(m)+deg(f) \geq 2*d_{OPT}$ thanks to the fact that we are going to remove nodes (so with minimual sum of degrees) in which the previous observation holds.
In $H_i$ there are $\frac{|H_i|}{2}$ pairs that will be removed until $i = 1$. There couples does not share nodes. The total cost of the edges in $H_i$ is greater than $\Big(2*d_{OPT}*\frac{|H_i|}{2}\Big)/2$ (the division by 2 is due to the fact that we want to avoid to consider edges twice). So we have that the density is greater than $\frac{2*d_{OPT}*\frac{|H_i|}{2}}{2*|H_i|} = \frac{d_{OPT}}{2}$.
Since the algorithm returns the graph with the highest density over all the iterations we have a solution with density at least $\frac{d_{OPT}}{2}$.
We proved that $michele\_party\_approx$ is a 2-approximation.

The cost is $O(n^3)$ because the most expensive operation in the body of the loop is the density that cost $|E|=n^2$.

\section{Exercise 2}

Our solution was inspired by the Set Cover approximation proof described in \cite{wikiset} but follows the path of the proof shown during the class.

Let $A$ is the set of required skills. $S$ is the set of all the people available, each people is represented as a set of skills $S_j \subseteq A$. Let $n = |A|$.

We can express the Set Cover with Redundancies problem using the following ILP formulation:

\[
    \begin{array}{ll}
    min \mathlarger{\sum}_{S_j \in S} c_j*x_j & \\
    \text{s.t.} \mathlarger{\sum}_{S_j | A_i \in S_j} x_j \geq 3 & \forall A_i \in A \\
    x_j \in \{0, 1\} & \forall S_j \in S \\
    \end{array}
\]

In order to build a randomized approximation consider the associated LP relaxion where $x^*_j \in [0, 1]$. The LP solution is a vector $x^*$ of real values.

Consider the algorithm ALG in which each person $S_j$ is chosen randomly with probability $p_j = min(d*log(n)*x^*_j, 1)$ with all choices that are indepedent.
We denote the vector of these choices as $x'$.

Let $C_i = \sum_{S_j | A_i \in S_j} x'_j$ a random variable that represents the times that the skill $A_i$ is covered.
The expectation of $C_i$ is $\E[C_i] = \E\Big[\sum_{S_j | A_i \in S_j} x'_j\Big] = \sum_{S_j | A_i \in S_j} p_j$.
Then, $\E[C_i] = d*log(n)*\sum_{S_j | A_i \in S_j} x^*_j \geq d*log(n)*3$ because the cover constraint in LP is satisfied.

From this, thanks to the Chernoff lower bound, follows that:

\bigskip
$Pr\Big[C_i < 3\Big] = Pr\Big[C_i < \Big(1 - \Big(1 - \frac{1}{d*log(n)}\Big)\Big)*d*log(n)*3\Big] \leq Pr\Big[C_i < \Big(1 - \Big(1 - \frac{1}{d}\Big)\Big)*d*log(n)*3\Big] \leq exp\bigg(-\frac{1}{2}*3*d*log(n)*\Big(1 - \frac{1}{d}\Big)^2\bigg)$
\bigskip

Note that $\Big(1 - \frac{d*log(n) -1}{d*log(n)}\Big)*d*log(n) = 1$. $0 < 1 - 1/d < 1$ must be valid in order apply Chernoff and so we have that $d > 1$.

We have that the probability that the skill $A_i$ is covered more than 3 times is $Pr[C_i \geq 3] \geq 1 - exp\bigg(-\frac{1}{2}*3*d*log(n)*\Big(1 - \frac{1}{d}\Big)^2\bigg)$.

%$\sum \E[x'_j] \leq \sum d*log(n)*c_j*x^*_j$. Using Markow inequality we can bound the objective function expected value $Pr[\sum x'_j \geq \sum 4*d*log(n)*x^*_j] \leq \frac{\sum d*log(n)*x^*_j}{\sum 4*d*log(n)*x^*_j} = \frac{1}{4}$.

The probability that at least one $A_i$ is not covered 3 times is $Pr\Big[\bigcup_{A_i \in A} C_i < 3\Big] \leq \mathlarger{\sum}_{A_i \in A} Pr\Big[C_i < 3\Big] = n*exp\bigg(-\frac{1}{2}*3*d*log(n)*\Big(1 - \frac{1}{d}\Big)^2\bigg)$.

Note that ncreasing $d$ give us a better probability that all skills are covered but also, on the other hand, decrease the probability to have a minimal solution because over a treshold we have that all the variables in $x'$ are 1.

The expected cost is $\E\big[\sum_{S_j \in S} c_j*x'_j\big] = \sum_{S_j \in S} c_j*\E[x'_j] = \sum_{S_j \in S} c_j*d*log(n)*x^*_j$ and so it is $d*log(n)$ times the cost of the LP.

With the Markov inequality we bound with a parameter $k$ the probability to fail on having an average cost less than $k*d*log(n)$ times the cost of LP.

\bigskip
$Pr\Big[\sum_{S_j \in S} c_j*x'_j \geq k*d*log(n)*\sum_{S_j \in S} c_j*x^*_j\Big] \leq
\frac{\mathlarger{\sum}_{S_j \in S} d*log(n)*c_j*x^*_j}{\mathlarger{\sum}_{S_j \in S} k*d*log(n)*c_j*x^*_j} = \frac{1}{k}$
\bigskip

Obviously this is an event that we want to avoid and setting a big $k$ help us in that but, on the contrary, give us also a worse approximation factor than $d*log(n)$.

We must choose $d$ and $k$ in a way that maximize the probability that our solution is valid and the expected cost is in the bound and maximize the probability that such solution is minimal.

Choosing $d = 3$ we have an interesting result:

$Pr\Big[\bigcup_{A_i \in A} C_i < 3\Big] = n*exp(-2*log(n)) = 1/n$

The probability that all the skills are covered, so that $x'$ is feasible, is $1 - 1/n$ that is quite high and so we expect to need only a run to get a feasible solution.

The cost of this approximation must be at most $3*k*log(n)*LP$.

The probability that the solution is not feasible or that the cost exceed the bound is $Pr\Big[\E\big[\sum_{S_j \in S} c_j*x'_j\big] \geq \sum_{S_j \in S} k*d*log(n)*x^*_j \lor \bigcup_{A_i \in A} C_i < 3\Big] \leq \frac{1}{k} + \frac{1}{n}$.

Setting $k = 2$ we get that the solution is feasible and with cost less than $6*log(n)*LP$ with probability greater than $\frac{1}{2} - \frac{1}{2*n}$ that is amost $\frac{1}{2}$ with a reasonable $n$.
We conclude that, as in the normal set cover approximation, the expected number of repetitions are $2$ and we choose $d$ and $k$ with the intention to have such number of repetitions.

\section{Exercise 3}

We denote as $F^*$ the optimal solution of the problem. $F^*$ is the minimum cost set of edges that if removed creates $k$ connected components with each target terminal $s_i$ inside each component.

Let $F_i^* \subset F^*$ the set of edges that if removed separated the vertex $s_i$ from the other $s_j$ with $i \ne j$. We have that $\mathlarger{\cup}_{i = 0}^k F_i^* = F^*$.

Each edge $e$ in $F^*$ is contained in two $F_i^*$ because it is incident at two connected components. So we can say that:

\[
\mathlarger{\sum}_{i = 0}^k \mathlarger{\sum}_{e \in F_i^*} w(e) = 2*\mathlarger{\sum}_{e \in F^*} w(e)
\]

Our algorithm returns $k$ min cuts $F_i$. By definition $F_i$ is the minimum set of edges that separates $s_i$ from the other $s_j$ with $i \ne j$ so we have that $\mathlarger{\sum}_{e \in F_i} w(e) \leq \mathlarger{\sum}_{e \in F_i^*} w(e)$.

This implies that:

\[
\mathlarger{\sum}_{i = 0}^k \mathlarger{\sum}_{e \in F_i} w(e) \leq 2*\mathlarger{\sum}_{e \in F^*} w(e)
\]

And so, in the worst case, the cost of $F$ is twice the cost of $F^*$. We proved that this is a 2-approximation algorithm.

\section{Exercise 4}

Let $k$ an ordered multiset of genes $g \in G$ such that $g_1 || ... || g_j || ... || g_{|k|} = D \text{ with } g_j \in k$ ($||$ is the concatenation operator). $K$ is the set of all possible $k$.

As instance, if $D = ACCA$ we can have $K = \{\{AC,CA\},\{ACC,A\}\}$ if $G = \{AC,CA,ACC,A\}$.

Let the variables $x_i$ with $i = 1 ... m$ binary variables that represents if a gene $G_i \in G$ is in $D$. 

Let $y_k$ $\forall k \in K$ a binary variable that tells if the multiset $k$ is used to form $D$.

The ILP formulation is the following:

\[
    \begin{array}{lll}
    & min \mathlarger{\sum}_{i = 1}^m w_i*x_i & \\
    (1) & \mathlarger{\sum}_{k \in K | G_i \in k} y_k - x_i \leq 0 & \forall i \in \{1 ... m\} \\
    (2) & \mathlarger{\sum}_{k \in K} y_k \geq 1 & \\
    & x_i \in \{0, 1\} & \forall i \in \{1 ... m\} \\
    & y_k \in \{0, 1\} & \forall k \in K
    \end{array}
\]

$(1)$ means that if a gene $G_i$ is not taken all the $k$ that contains $G_i$ mus be excluded. $(2)$ means that at least one $k$ must be taken, and it is our solution that has minimum cost genes.

Consider now the LP-relaxion with $x_i \geq 0 \land y_k \geq 0$. Let's compute the dual of it.
In the dual, we have that the number of variables is the same of the number of constraints in the primal ($m+1$) and vice-versa the number of constraints is the number of variables in the primal ($m+|K|$).
Only constraint $(2)$ contribute to the objective function (cfr. \cite{cap8roma}).

When $G_i$ is used the index $i$ is always relative to $G$ and when we use $G_i \in k$ we are considering all the distinct genes in $k$.

\[
    \begin{array}{ll}
    max\text{ }u & \\
    \mathlarger{\sum}_{G_i \in k} v_i - u \geq 0 & \forall k \in K \\
    v_i \leq w_i & \forall i \in \{1 ... m\}\\
    u \geq 0, v_i \geq 0 & \forall i \in \{1 ... m\}
    \end{array}
\]

\section{Exercise 5}

The proposed problem is a finite zero-sum game theory problem.

\begin{center}
    \begin{tabular}{ | l | l | l | }
        \hline
        Comet, Dasher & Head & Tail \\ \hline
        Head & 4, -4 & -1, 1 \\ \hline
        Tail & -2, 2 & 2, -2 \\ \hline
    \end{tabular}
\end{center}

We create a matrix $A_{2,2}$ associated to the previous table taking the first value of each table entry so that each entry of the matrix is the payoff for Comet and the cost for Dasher.
We encode payoff/cost and mixed strategy as variables in a LP problem for each player:

\[
    \begin{array}{lll}
    \text{Comet} & \text{  } & \text{Dasher} \\
    max\text{ }y                & & min\text{ }v \\
    4*x_1 - 2*x_2 \geq y    & & 4*u_1 - u_2 \leq v \\
    -x_1 + 2*x_2 \geq y     & & -2*u_1 + 2*u_2 \leq v \\
    x_1 + x_2 = 1               & & u_1 + u_2 = 1 \\
    x_1 \geq 0, x_2 \geq 0      & & u_1 \geq 0, u_2 \geq 0 \\
    \end{array}
\]

Comet wants to maximize the payoff choosing the mixed strategy $x$ in a way that no matter what Dhaser plays. Dasher wants to minimize the cost in the same way.
Each problem is the dual of the other so $max\text{ }y = min\text{ }v = k$. This implies that this is a Mixed Nash Equilibrium with $k$ that is the value of the zero-sum game.

Solving the two problems with a solver (see the \hyperref[ex5app]{Appendix} to know how) we have that $max\text{ }y = min\text{ }v = \frac{2}{3}$, $x_1 = \frac{4}{9}$, $x_2 = \frac{5}{9}$, $y_1 = \frac{1}{3}$, $x_2 = \frac{2}{3}$.
Decoding the index of the variables to the strategies the result is that Santa will bias the coins with this probabilities:

\[
    \begin{array}{ll}
    Pr[\text{Comet coin = Head}] = \frac{4}{9} & Pr[\text{Comet coin = Tail}] = \frac{5}{9} \\
    Pr[\text{Dasher coin = Head}] = \frac{1}{3} & Pr[\text{Dasher coin = Tail}] = \frac{2}{3} \\
    \end{array}
\]

\section{Exercise 6}

Let $h$ the position of Giorgio's home.
Let $S(i) = Pr[Safe|x(t)=i]$ the probability that Giorgio goes to home safely when he is at position $i$. We know that $Pr[x(t+1) = x(t)+1] = p$ and $Pr[x(t+1) = x(t)-1] = 1-p$. Follows that:

\[
S(i) = \begin{cases}
0 & \text{if }i = -1 \\
1 & \text{if }i = h \\
p*S(i-1) + (1-p)*S(i+1) & \text{otherwise} \\
\end{cases}
\]

We have that in general $S(i) = p*S(i-1) + (1-p)*S(i+1)$ and $S(i) = p*S(i) + (1-p)*S(i)$ (from $1 = p - (1-p)$) and so $S(i+1) = \frac{p}{1-p}*\big(S(i) - S(i-1)\big) + S(i)$.
Follows that $S(i+2) = \frac{p}{1-p}*\big(S(i+1) - S(i)\big) + S(i+1) = \frac{p}{1-p}\Big(\frac{p}{1-p}\big(S(i) - S(i-1)\big) + S(i) - S(i)\Big) + \frac{p}{1-p}*\big(S(i) - S(i-1)\big) + S(i)$.

With $i = 0$ we have $S(2) = \big(\frac{p}{1-p})^2*S(0) + \frac{p}{1-p}*S(0) + S(0)$ and so clearly:

\[
S(i) = \sum_{j=0}^{i} \Big(\frac{p}{1-p}\Big)^j*S(0).
\]

Consider now when $i = h$. The problem says that Giorgio makes an infinite number of steps so we can set $h = +\infty$.
\[
1 = S(+\infty) = S(0)*\sum_{j=0}^{+\infty} \Big(\frac{p}{1-p}\Big)^j.
\]
This is a geometric series \cite{geo}. By definition if $|\frac{p}{1-p}| < 1$ it converges to $\frac{1}{1-\frac{p}{1-p}}$.

The probability that Giorgio goes to the hospital from position 0 is $Pr[Hospital|x(t)=0] = 1 + Pr[Safe|x(t)=0] = 1 - S(0)$.

$|\frac{p}{1-p}| < 1 \Leftrightarrow p < 1/2$ and so, finally, we have that:
\[
Pr[Hospital|x(t)=0] = 1 - \frac{1}{\sum_{j=0}^{+\infty} \big(\frac{p}{1-p}\big)^j} = \begin{cases}
\frac{p}{1-p} & p < 1/2 \\
1 & p \geq 1/2 \\
\end{cases}
\]

Giorgio goes to hospital for sure when $p \geq \frac{1}{2}$ and goes to hospital with probability at most $\frac{1}{2}$ when $\frac{p}{1-p} \leq \frac{1}{2} \Rightarrow p \leq \frac{1}{3}$.

\newpage
\section*{Appendix}

\subsection*{Exercise 5}
\label{ex5app}

Yes, AMPL+CPLEX should be used insted of Z3 but we do not like the AMPL syntax and in addition we use Z3 almost every day for SMT/SAT so...

\bigskip
Comet:
\begin{verbatim}
(declare-fun y () Real)
(declare-fun x2 () Real)
(declare-fun x1 () Real)
(assert (>= (- (* 4.0 x1) (* 2.0 x2)) y))
(assert (>= (+ (- x1) (* 2.0 x2)) y))
(assert (= (+ x1 x2) 1.0))
(assert (>= x1 0.0))
(assert (>= x2 0.0))
(maximize y)
(check-sat)
\end{verbatim}

Dasher:
\begin{verbatim}
(declare-fun v () Real)
(declare-fun y2 () Real)
(declare-fun y1 () Real)
(assert (<= (- (* 4.0 y1) y2) v))
(assert (<= (+ (* (- 2.0) y1) (* 2.0 y2)) v))
(assert (= (+ y1 y2) 1.0))
(assert (>= y1 0.0))
(assert (>= y2 0.0))
(minimize v)
(check-sat)
\end{verbatim}


\vfill
\bibliography{main} 
\bibliographystyle{ieeetr}

\end{document}
